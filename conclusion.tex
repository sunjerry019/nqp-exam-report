\section{Conclusion and Areas for Improvement}
    In conclusion, we solved the eigenvalue problem for a system size up to $L = 11$ ($12$ lattice sites in total) in the time that we were given. Unfortunately, on the last day, we noticed an error in the code that meant that we had to rerun all the calculations. Due to time constraints, we were only able to solve for a system size up to $L = 10$ ($11$ lattice sites in total).

    Of course, given the short timeframe, the work done in this project could be improved in many ways, on top of what we have already mentioned within the text. 
    
    Firstly, the condensate fraction calculation could have been integrated with the spectrum calculation so that the same Hamiltonian need not be diagonalised twice. 
    
    Secondly, The \texttt{run.py} script could also have recorded the runtimes of each of the steps, which would have allowed us to visualise the scaling of the problem as a function of lattice size $L$. 

    Thirdly, the exact reason why the code does not run on the cluster could be further investigated so that one may have access to much larger computational resources.

    Last but not least, we could have also employed the Jordan-Wigner Transformation to reduce the complexity of the problem numerically \cite{wilkeSymmetryprotectedBoseEinsteinCondensation2022}. 

    

    